\documentclass[]{article}
\usepackage[utf8]{inputenc}
\usepackage[ngerman]{babel}
\usepackage{textcomp}
\usepackage{hyperref}

\usepackage{listings}
\lstset{
	language=C++,
	tabsize=4,
	keepspaces,
	extendedchars=true,
	rulecolor=\color{black},
	basicstyle=\footnotesize,
	aboveskip=5pt,
	upquote=true,
	columns=fixed,
	showstringspaces=false,
	extendedchars=true,
	breaklines=true,
	frame=single,
	showtabs=true,
	showspaces=false,
	showstringspaces=false,
	basicstyle=\tiny,
	keywordstyle=\color{blue}
}

%opening
\title{First steps}
\author{Wombacher Sascha}

\begin{document}
\maketitle


\section{Introduction}

This small guide will help you install and run all provided examples and provide some info about how to add you own projects.

\mbox{} \\
\textbf{Getting started:}
\begin{itemize}
	\item Introduction for Windows users (section \ref{Windows})
	\item Introduction for \space macOS \space \space users (section \ref{Mac})
	\item Introduction for \space \space Linux \space \space \space users (section \ref{Linux})
\end{itemize}

\mbox{} \\
\textbf{Available documentation} (located in \textit{$<$ProjectDirectory$>$/Documentation}):
\begin{itemize}
\item HTML (recommended)
\item PDF
\end{itemize}

\section{General}
This library uses some third Party libraries:
\begin{itemize}
	\item GLM (GL mathematics for vector and matrix operations)
	\item OpenCV (ComputerVision library for 2D graphics)
	\item Glut/Freeglut (3D, currently only OpenGL 1.x is used)
	\item InfInt (Lib for very long int values, C\# equivalent BigInt)
\end{itemize}

\newpage
\section{Windows}\label{Windows}

\textit{\textbf{Installation:}}
\begin{enumerate}
	\item Install Visual Studio 2015/2017, link: \url{https://www.visualstudio.com/downloads/}
	\begin{itemize}
		\item Required should only be the default C++ development tools
		\item if this doesn't work: Install all packages which are mentioning 'C++'
	\end{itemize}
	\item Run 'createVisualStudioSolution\_$<$Version$>$.bat'
	\item Open Visual Studio solution $<$ProjectDirectory$>$/VisualStudio/ALL\_BUILD
	\item Build all (Debug/Release libraries should both be linked)
	\item Set a example Project: Right click on example $\rightarrow$ set as start Project
	\item Run example 
\end{enumerate}

\mbox{} \\
\textit{\textbf{Create a new project:}}
\begin{enumerate}
\item Add a new *.cpp file in directory 'Solutions'
\item Run 'createVisualStudioSolution\_$<$Version$>$.bat'
\end{enumerate}

\mbox{} \\
\textit{\textbf{Remove a project:}}
\begin{enumerate}
\item Remove the corresponding *.cpp file from the 'Solutions' directory
\item Run 'createVisualStudioSolution\_$<$Version$>$.bat' \\
\end{enumerate}

\subsection{Some helpful Visual Studio functionalities \\}

\underline{\textbf{Set a command line parameter}} \\

Some Examples require a command line argument(s). If you want to use those parameter you have three options:
\begin{itemize}
	\item Set parameter in Visual studio:
	\begin{itemize}
		\item Right click on project $\rightarrow$ preferences
		\item Go to Debugging $\rightarrow$ command arguments
	\end{itemize}
	\item Open program with command line and add a parameter
	\item If you only have 1 input parameter you can drag and drop a file onto the executable
\end{itemize}


\newpage
\section{Mac}\label{Mac}

\textit{\textbf{Installation:}}
\begin{enumerate}
	\item Install XCode, link: \url{https://developer.apple.com/download/}
	\item Install dependencies by running 'mac\_installDependencies.sh'
	\item Run 'createXCodeProject.sh'
	\item Open XCodeProject: $<$ProjectDirectory$>$/XCodeProject/*.xcodeproj
	\item Build all examples
	\item Choose a example and run it 
\end{enumerate}

\mbox{} \\
\textit{\textbf{Create a new project:}}
\begin{enumerate}
\item Add a new *.cpp file in directory 'Solutions'
\item Run 'createXCodeProject.sh' 
\end{enumerate}

\mbox{} \\
\textit{\textbf{Remove a project:}}
\begin{enumerate}
\item Remove the corresponding *.cpp file from the 'Solutions' directory
\item Run 'createXCodeProject.sh' 
\end{enumerate}

\newpage
\section{Linux}\label{Linux}
Currently three dependency commands are available:
\begin{itemize}
	\item Apt-Get based (Ubuntu/Debian/...) 
	\item Pacman based (Arch/Manjaro/...)
	\item Yum based (Fedora/RedHat/...) 
\end{itemize}

\mbox{} \\
If you own another distribution you should know how to install dependencies, required packages:
\begin{itemize}
	\item GL
	\item GLU
	\item Freeglut
	\item GLM
	\item OpenCV (2.4 and 3.1 have been tested)
	\item libxmu 
	\item libxi
	\item cmake 
\end{itemize}

\subsection{Linux installation}

\textit{\textbf{Installation:}}
\begin{enumerate}
	\item Install dependencies by running 'linux\_installDependencies.sh'
	\item Run 'createMakeFileProject.sh'
	\item Run 'cd $<$ProjectDirectory$>$/MakeFileProject' $\rightarrow$ 'make -j 4'
	\item All compiled executables are located in: $<$ProjectDirectory$>$/bin
	\item Choose a example and run it 
\end{enumerate}

\mbox{} \\
\textit{\textbf{Create a new project:}}
\begin{enumerate}
\item Add a new *.cpp file in directory 'Solutions'
\item Run 'createMakeFileProject.sh' 
\end{enumerate}

\mbox{} \\
\textit{\textbf{Remove a project:}}
\begin{enumerate}
\item Remove the corresponding *.cpp file from the 'Solutions' directory
\item Run 'createMakeFileProject.sh' 
\end{enumerate}

\subsection{Linux IDEs}

There are a couple of IDE's available for Linux. Possible IDE's with integrated cmake support are: 
\begin{itemize}
\item QtCreator
\item CLion (free for students)
\end{itemize}

\end{document}
